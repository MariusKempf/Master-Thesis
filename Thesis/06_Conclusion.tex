\section{Conclusion \& Outlook}
\label{sec:ConclusionOutlook}

\subsection{Conclusion}
\label{subsec:Conclusion}
% intro
In order to use deep learning models in clinics, they must be robust and generalize sufficiently. One way to achieve this is to train these models on very large datasets. However, annotated image data in the healthcare domain is rare and expensive---and is subject to privacy aspects.
Federated learning is a promising approach to address these challenges. But first, technical solutions are needed to enable the application of federated learning in the healthcare environment.

% We compared exisiting federated learning solutions with regrad to what is needed bring federated learnign into medical institutions. Thus, we provide an overview of solutions in the 
% there is not yet one solution that meets all requirements, but promising solutions are emerging.

% Solution comparison
We showed which federated learning solutions exist and what is needed to bring federated learning into medical institutions. There is not yet one solution that meets all requirements, but many promising projects could be identified.
% on federated learning and also its application within the healthcare environment.
The platform solutions NVIDIA Clara Federated and JIP Federated, the latter of which is presented within this article, provide features that are advantageous for the application in the healthcare environment, in particular for the usage of medical images.

We demonstrated that the Joint Imaging Platform is a promising solution, either in combination with an existing federated learning framework or as a stand-alone solution itself.
% Integration
With the seamless integration of PySyft, the high flexibility of the platform was demonstrated, so the possibility to combine other solutions with the Joint Imaging Platform's infrastructure and the resulting benefits are given.
% JIP Federated
Further, we designed a JIP-only federated learning solution---JIP Federated.
By using JIP Federated for the segmentation of gliomas in brain MRI scans, we demonstrated that it is a capable federated learning solution for complex medical task. Its implementation is highly flexible, easy to apply on other federated learning experiments, and brings further advantages through its platform features.
Thus, we provide a comprehensive open-source solution to conduct real-world FL in the domain of medical image computing.

% ### Outlook ###
% Advantage: Established & Trusted
The Joint Imaging Platform is an established and trusted tool in the healthcare community and is already being used in a multitude of German clinics and beyond.
As proved by the experiments, JIP Federated can be a solution to make federated learning real-world.
% real-world experiment with lots of data from 
Future work should focus on using JIP Federated for experiments across  geographically distributed clinics on large datasets that cannot be shared directly.
% Adding additional privacy mechanisms and layer between the communication of JIP instances
A further aspect needs to be the extension of the Joint Imaging Platform to provide privacy mechanisms such as Differential Privacy, Homomorphic Encryption, and Secure Multi-Party Computing as additional security layers for the communication between instances.
% Motivation zur Anwendung in der Praxis tatsächlich ohen den Zugang zu den ganzen Daten.
With this work on federated learning with the Joint Imaging Platform, we also hope to give an incentive for further federated experiments using JIP Federated.
% We want to encourage the further development and long-term use of JIP Federated in practice and for large-scale studies.

\subsection{Outlook}
\label{subsec:Outlook}

Medical image analysis has come a long way from rule-based systems and handcrafted features for early computer vision to modern techniques. Nowadays, deep learning is the technology of choice in computer vision and therefore also for medical images. In recent time, it has proved its capabilities and potential on a multitude of tasks in the domain of medical images. Deep learning is currently and will be in the future the technology for processing and analyzing medical images. However, its full potential can only be unlocked if the appropriate volume of data is available to train such deep architectures.

% federated learning
Federated learning provides a solution to overcome the challenge of having a sufficient amount of data. If the gap between simulation and real application can be closed by developing comprehensive technical solutions, enormous volumes of data will be available for usage. All around the world, respective data is collected on a daily basis, but until now it has not been available for traditional machine learning settings, also for reasons of patients privacy and regulations. This data would be accessible and usable by means of technical solutions for federated learning across geographically distributed institutions.

% results
Using that data would result in among others two consequences.
The developed models can achieve significantly higher performance on numerous tasks, which is important in a critical field like medicine where life and death can be at stake.
Further, these models will be able to generalize better because of their extensive knowledge base. That is, they are more capable to adapt to previously unseen samples which do not fall into the distribution of the samples used for training. 
All of this would be possible, because systems are available that have gained experience from the data of millions of patients and the knowledge of numerous medical experts who labeled the data.
% improve health-care service
Such comprehensive systems can contribute to improve healthcare, individualize patient treatment, and to better understand or even cure severe and rare diseases.
% remote areas 
Further, models with high performance and the ability to generalize sufficiently could even be used in regions of the world with poor or hardly any advanced healthcare services.

% final
Certainly, many things remain to be done to bring the technical solutions to the point where federated learning projects truly go across medical institutions, but first steps are being taken. The healthcare sector remains very sensitive, highly regulated and also political, but there are more than just a few researches who see federated learning as the future of digital health \citep{Rieke2020TheLearning}.


% BACKUP

%Further, one can imagine that models with high performance and the ability to generalize sufficiently can actually be developed by using the data volumes that will be available with real-world federated learning. The resulting solutions and services could even be used in regions of the world with only poor or hardly any healthcare services.
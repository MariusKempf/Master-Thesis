{\Large \textbf{Abstract}} 

\bigskip

%Motivation
Data science offers many opportunities to gain new and valuable insights from medical data to achieve enhanced patient care. The potential is immense, but medical (imaging) data is sensitive and no public resource. This often conflicts with the demands of data hungry machine learning techniques. Federated learning is a paradigm that promises to leverage geographically distributed data while maintaining privacy constraints. However, while the principle methodology for federated training mechanisms is readily established on basis of simulated scenarios, real-world applications in the health care sector are still rare and come with plenty of remaining challenges.
% Methods
We, therefore, ask the question, what is needed to bring a federated learning solution into clinics.
To bridge this gap in the medical imaging domain, we gathered requirements for respective technical solutions for real-world federated learning. We propose an integration of PySyft into the Joint Imaging Platform and a solution purely based on the Joint Imaging Platform: JIP Federated. The proposed as well as several existing federated learning solutions are evaluated using the identified requirements.
% Results 
We demonstrate the applicability of our solution on basis of a federated segmentation experiment on distributed brain MRI scans. JIP Federated is flexible and able to address some of the challenges that the community currently faces. It is available as an open-source project.
% Conclusion
The proposed solution might contribute to facilitating applied federated learning projects that truly go across medical institutions.







\begin{comment}
% OLD Abstract version (without feedback from KMH)
% intro
Data science offers many opportunities to gain new and valuable insights from medical data to achieve enhanced patient care. The potential becomes even greater the more data can be analyzed and used to train machine learning models. In particular, for deep learning, numerous training samples are needed to achieve adequate performance and generalizability for the sensitive healthcare domain. However, medical data is highly sensitive and must therefore be treated very carefully for good reasons.
% question
Federated learning is a machine learning paradigm that allows to leverage geographically distributed data while maintaining privacy by sharing models instead of raw data during the training process.
The potential and performance of models trained federally have already been demonstrated, but real-world applications in the field of medical imaging are still rare. To bridge the gap between simulations and actual experiments across medical institutions, technical solutions are required.
Recently, a multitude of promising projects aiming to provide solutions for federated learning have emerged. However, they must fulfill specific requirements for a meaningful usage in a medical environment.
We, therefore, ask the question, what is needed to bring federated learning solutions into clinics and how a solution could look like.
% methodology
We compare existing solutions and demonstrate a combination of PySyft and the Joint Imaging Platform.
Further, a solution purely based on the Joint Imaging Platform, which is already installed at several German clinics, is introduced: JIP Federated. Its capability is demonstrated with a federated segmentation experiment on distributed brain MRI scans.
% results
Our results show the following.
The gap of a comprehensive solution fulfilling all requirements could be narrowed to some extent, but not yet closed---Additional effort is required.
Further, JIP Federated is a flexible solution that allows data scientists to freely implement their own algorithms, and is available as an open-source project. 
Together with the Joint Imaging Platform's features developed for medical imaging, JIP Federated offers a promising solution to apply federated learning across medical institutions, thus, allowing distributed analyses while preserving privacy.
\end{comment}

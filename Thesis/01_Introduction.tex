\section{Introduction}
\label{sec:Introduction}


% Einstieg: Federated Learning for Thyroid Ultrasound Image Analysis to Protect Personal Information: Validation Study in a Real Health Care Environment
% Einstieg: Preserving Patient Privacy while Training a Predictive Model of In-hospital Mortality

The recent years have proved that deep learning (DL) \citep{Lecun2015DeepLearning} is able to achieve state-of-the-art performance, even surpassing human-level performance, on a multitude of different tasks ranging from computer vision and speech recognition to natural language processing \citep{He2015DelvingClassification, Hinton2012DeepGroups, Devlin2019BERT:Understanding}. Also, within the medical domain, deep learning has gained attention, and models with expert-level performance have been reported \citep{Esteva2017Dermatologist-levelNetworks}.

% amount of data
The availability of huge amounts of annotated data is one of the key factors for this success, namely datasets such as ImageNet \citep{Deng2009ImageNet:Database}. Deep learning in the medical domain is not different in that regard: To achieve human-level or clinical-grade performance, the amount and quality of annotated data available for model development are crucial. This applies in particular to very rare diseases or diseases with highly heterogeneous disorders \citep{Esteva2017Dermatologist-levelNetworks, DeFauw2018ClinicallyDisease, Dluhos2017Multi-centerApproach}.
%(such as schizophrenia)

% Privacy & Barriers
When privacy is not a concern, the available data is usually centralized at one location and is then used for model training. Often, the datasets are even made publicly available as a contribution to the scientific community. In the medical domain, however, data is highly personal and, therefore, sensitive in terms of patients' privacy. Conclusively, its usage is strictly regulated and comes with a multitude of barriers and regulations, such as the United States \textit{Health Insurance Portability and Accountability Act} (HIPPA \cite{USDepartmentofHealthandHumanServices2020HealthHIPPA}) or the European \textit{General Data Protection Regulation} (GDPR\cite{IntersoftConsulting2016GeneralGDPR}) \citep{vanPanhuis2014AHealth}.

% Federated Learning
A paradigm to address and overcome the challenges of data volume and privacy while still allowing to learn over multiple, distributed instances, is Federated Learning (FL). Federated learning decouples the model training from the need for direct access to the raw training data of each participant. Its collaborative approach of aggregating knowledge from models trained in a decentralized manner works without accessing or exchanging the underlying data \citep{BrendanMcMahan2017}.

% Application in Health care
Initially developed for learning from private data on mobile devices, such as written text, it is also applicable within the medical domain. One can think of a multitude of different applications of federated learning in healthcare \citep{Xu2021}. For instance, by applying federated learning on distributed electronic health records, a Support Vector Machine, Multi-layer Perceptron, and Logistic Regression model have been trained to predict adverse drug reactions of patients and have achieved comparable performance metrics to models trained on centralized data \citep{Choudhury2019PredictingLearning}. It has also been shown that federated learning can be used to improve models predicting the in-hospital mortality of patients admitted to the intensive care unit \citep{Sharma2019PreservingMortality}.

% FL medical images
In recent times, an increasing number of publications address federated learning in the context of medical image data and therefore combine federated learning with deep learning.
For instance, Kaissis et al. applied federated deep learning on paediatric chest x-ray images, showing classification performance similar to locally trained models while maintaining patient privacy \citep{Kaissis2021End-to-endImaging}.
Sheller et al. trained a model for tumor segmentation with federated learning, simulating a real distribution across institutions by separating  magnetic resonance image (MRI) scans of subjects with brain tumors according to the meta-information about the origin of the images. The authors showed that 99\% of the model performance compared to training on centralized data is reached \citep{Sheller2020FederatedData}.
\cite{Roth2020FederatedImplementation} showed that models trained in a federated manner perform on average 6.3\% better than their counterparts trained on the locally available data only. The objective was to classify breast density based on mammographic images. Their work has a distinctive characteristic: federated learning was applied in a real-world scenario.
While many publications report federated learning experiments in a simulated setting, only few performed experiments across geographically distributed institutions (see \Cref{subsec:LitRev}).

% Problem Statement
% Contextualize the problem -  Now: real-world solutions needed
To fully profit from federated learning in healthcare, technical solutions for its real-world application need to be identified. Currently, there is a lot of activity resulting in promising projects which address this topic. Frameworks such as \textit{PySyft} by \cite{Ryffel2018ALearning} or platform solutions like \textit{NVIDIA Clara Federated}\footnote{\url{https://developer.nvidia.com/clara}} have attracted attention in the research community and are contributing to make federated learning real-world. While these federated learning solutions are tackling the challenges with available but private data, the healthcare sector is a very specific environment. Therefore, a possible application in real clinics comes with specific requirements to the eligible solutions.

% Set your aims and objectives - requirements / comparison
% The aim of this research is to investigate %
This work investigates which functionalities are provided by current FL solutions and what technical requirements are necessary to bridge the gap between FL solutions and their application in medical institutions. 
% implementation & experiments
Further, we demonstrate how the Joint Imaging Platform\footnote{\url{https://jip.dktk.dkfz.de/jiphomepage/}} (JIP) by \cite{Scherer2020JointAnalytics} can be used for FL in two ways.
Due to the JIP's openness and flexibility, it can be easily combined with other solutions. Therefore, one option is to integrate another FL solution into the JIP, as we demonstrate with the integration of PySyft. Further, the already existing components of the JIP can be used to realize FL - this is what we introduce as JIP Federated. For both options, the implementations are described in detail. 
In addition, we conducted experiments to measure the runtime of the JIP-based solutions, as well as the performance of JIP Federated when segmenting gliomas in brain MRI scans.
